\documentclass{article}

\begin{document}

  \begin{titlepage}
    \centering
    \vspace*{2cm}
    
    \Huge
    \textbf{Simulador de Polarización en Redes}
    
    \vspace{1.5cm}
    
    \Large
    Jhorman Gomez{\textsuperscript{1}}, Ivan Ausecha{\textsuperscript{2}}, James Calero{\textsuperscript{3}}, Daniel Rojas{\textsubscript{4}}
    
    \vspace{0.5cm}
    
    \large
    2326867{\textsuperscript{1}}, -{\textsuperscript{2}}, 2243461{\textsuperscript{3}}, -{\textsuperscript{4}}
    
    \vspace{0.5cm}
    
    \Large
    Universidad del Valle
    
    \vspace{0.5cm}
    
    \large
    Facultad de Ingeniería
    
    \vspace{0.5cm}
    
    \large
    Escuela de Ingeniería de Sistemas y Computación
    
    \vspace{0.5cm}
    
    \large
    Santiago de Cali, Octubre de 2024
    
  \end{titlepage}

  \section{Técnicas y Estructuras de Datos Utilizadas}
    
    \subsection{min\_p}
    La función \textbf{min\_p} Tiene como objetivo encontrar un punto p ∈ [min, máx] tal que f(p) sea mínimo, bajo la suposición de que f es convexa. El algoritmo realiza una búsqueda secuencial en el intervalo y ajusta progresivamente el rango de búsqueda hasta encontrar el mínimo de f(p).

    \begin{itemize}
      \item \textbf{Recursión:} genera un proceso de recursión lineal mediante el cual por cada iteración se aproxima más al punto mínimo de la función recibida.
      \item \textbf{Funciones de Alto Orden:} recibe una función como parámetro.
      \item \textbf{Colecciones:} genera un rango de valores a partir del min y max recibidos, a este rango se le aplica un map y al resultado del map se le aplica el método minBy, ambos métodos pertenecientes a colecciones.
    \end{itemize}

    \subsection{rhoCMT\_Gen}
        La función \textbf{rhoCMT\_Gen} hace uso de:

    \begin{itemize}
      \item \textbf{Funciones de Alto Orden:} hace uso de funciones anónimas y retorna esta misma como resultado.
      \item \textbf{Colecciones:} hace uso del método zip para generar una colección de tuplas (Frecuency, DistributionValue) sobre la cual se define la función sum.
    \end{itemize}

    \subsection{normalizar}
    La función \textbf{normalizar} Tiene como objetivo normalizar una medida de polarización para que se exprese en relación con un "caso peor" teórico. La normalización se logra dividiendo la polarización calculada para una distribución dada entre la polarización calculada para el caso extremo de polarización máxima. Esto permite una comparación consistente de la polarización entre distribuciones, sin depender de la escala absoluta.
    La función \textbf{normalizar} recibe una medida de polarización m, que se calcula para una distribución dada distribution. La polarización para esta distribución se calcula como: polarización real = m(frecuencias, valores de la distribución)
    A continuación, se calcula la polarización en el "peor caso", que corresponde a una distribución con los valores de frecuencia [0.5,0, . . . ,0,0.5], lo cual representa una distribución de creencias completamente polarizada en dos extremos: polarización peor caso = m(frecuencias peor caso, valores de la distribución)
    Finalmente, la polarización normalizada se calcula como: polarización normalizada = polarización real / polarización peor caso
    En conclusión, la función \textbf{normalizar} está correctamente definida y proporciona una medida relativa de la polarización de una distribución en relación con un caso extremo de polarización máxima. La implementación garantiza una precisión adecuada y se basa en una normalización coherente con la teoría de la polarización.

    \begin{itemize}
      \item \textbf{Funciones de Alto Orden:} hace uso de funciones anónimas y retorna esta misma como resultado.
      \item \textbf{Colecciones:} hace uso de vectores y en especial el método fill para generar una colección con la mayor polarización posible para una frecuencia de longitud k.
    \end{itemize}

    \subsection{rho}
    La función \textbf{rho} hace uso de:

    \begin{itemize}
      \item \textbf{Funciones de Alto Orden:} hace uso de funciones anónimas y retorna esta misma como resultado.
      \item \textbf{Colecciones:} hace uso de colecciones y métodos característicos de estas para generar los intervalos sobre los cuales se van a clasificar los agentes y a su vez estos se clasifican haciendo uso de métodos como map y groupBy.
      \item \textbf{Iteradores:} hace uso de métodos que requieren de iteradores como ZipWithIndex y indexWhere.
      \item \textbf{Reconocimiento de Patrones:} hace uso de reconocimiento de patrones para el reconocimiento de los elementos que hacen parte de las colecciones utilizadas, esto con el fin de poder hacer transformaciones sobre dichas colecciones y en últimas clasificar los agentes y obtener la frecuencia final a partir de las opiniones de estos.
    \end{itemize}

    \subsection{confBiasUpdate}
    La función \textbf{confBiasUpdate} hace uso de:

    \begin{itemize}
      \item \textbf{Funciones de Alto Orden:} recibe funciones como parámetro, en específico la función swg que corresponderia a la matriz de influencia.
      \item \textbf{Colecciones:} recibe una colección y en base a esta mediante el uso de map, se define la función sum sobre la que se va a realizar el respectivo update a las creencias de los agentes haciendo uso de un rango y mapeando dicha función para cada valor del rango.
    \end{itemize}

    \subsection{simulate}
    La función \textbf{simulate} Simula la evolución de las creencias de los agentes durante un número de pasos de tiempo t.
    La función \textbf{simulate} toma como entrada una función de actualización fu (como confBiasUpdate), un gráfico ponderado swg, las creencias iniciales b0, y el número de pasos de tiempo t.
    En cada paso de la simulación, se actualizan las creencias de los agentes usando la función de actualización fu.
    Formalmente, podemos escribir la definición recursiva como:
    simulate(fu,swg,b_0 )={█(b_0                               si t=0 }
                          {simulate(fu,swg,〖fu(swg,b〗_0),t-1) si t>0)}

    \textbf{Correctitud:}
    Recursión: La función \textbf{simulate} se implementa de manera recursiva. En cada llamada recursiva, las creencias actuales se actualizan utilizando la función fu, y luego se hace una llamada recursiva con las nuevas creencias.
    Base de la recursión: La base de la recursión ocurre cuando t == 0, en cuyo caso se retorna un vector con las creencias iniciales b0.
    Propiedad de terminación: El algoritmo termina correctamente después de t pasos, ya que cada llamada recursiva reduce el valor de t en 1. La función finalmente retornará una secuencia de creencias después de t pasos de simulación.
    Conclusión: La función \textbf{simulate} es formalmente correcta, ya que implementa una simulación recursiva bien definida de la evolución de las creencias, y termina correctamente después de t pasos.

    \begin{itemize}
      \item \textbf{Funciones de Alto Orden:} recibe funciones como parámetro, en específico las funciones correspondientes a la matriz de influencia (swg) y la función sobre la cual se van a actualizar las creencias de los agentes (fg).
      \item \textbf{Colecciones:} recibe una colección la cual se actualiza en base a fg y mediante estas actualizaciones se va creando una secuencia indexada donde cada elemento es la creencia de los agentes en el tiempo t empezando desde t=0.
      \item \textbf{Recursión:} genera un proceso de recursión lineal sobre el cual se actualiza t-1 veces la creencia recibida (sb).
    \end{itemize}

  \section{Informe de Corrección}

    \subsection{min\_p}
    La corrección de la función \textbf{min\_p} se argumenta por la propiedad de que, dado que f es convexa, la función es decreciente en una mitad del intervalo y creciente en la otra. Al dividir el intervalo y evaluar f en los puntos medios, se puede identificar cuál subintervalo contiene el mínimo.
    la funcion \textbf{min\_p} realiza un proceso recursivo. En cada paso recursivo, la función divide el intervalo en subintervalos y busca el mínimo en el subintervalo con el valor mínimo de f(p). Debido a la convexidad de f(p), el valor mínimo se encuentra en uno de los extremos del subintervalo o en un punto donde la derivada se anule.
    El proceso recursivo termina cuando el intervalo [min, max] es suficientemente pequeño, es decir, cuando |máx - min| < prec. En este caso, min_p retorna el punto medio del intervalo. Dado que la función es convexa, el mínimo estará cerca de este punto medio, por lo que el valor retornado es una aproximación adecuada.

    \subsection{rhoCMT\_Gen}

    \subsection{rho}

    \subsection{confBiasUpdate}

    \subsection{simulate}

  \section{Técnicas de Paralelización y Análisis de Impacto}

    \subsection{Tipos de Paralelización Usados}
    Para las funciones \textbf{rhoPar} y \textbf{confBiasUpdatePar} se utilizó solamente paralelismo de datos. A continuación  las razones:

    \begin{itemize}
      \item \textbf{rhoPar:} no hicimos uso de paralelismo de tareas porque abstracciones como \textbf{parallel} son solo útiles cuando se realizan procesos recursivos, además, la abstracción \textbf{task} realmente no tiene cabida dentro de nuestro algoritmo debido a que no es posible por ejemplo ir clasificando las creencias de los agentes antes de que todos los intervalos estén definidos.
      \item \textbf{confBiasUpdatePar:} a la hora de realizar pruebas comparando dos versiones de confBiasUpdate, la primera donde solo se hace uso de paralelismo de datos y la segunda donde se hace uso de tanto paralelismo de datos como de tareas. Nos encontramos con que en general la media obtenida para la primera versión es mejor que la media de la segunda, a esto le sumamos que hubo ocasiones en las que a la hora de realizar pruebas para la segunda versión hubo stackOverFlow, es decir, se solicitaba más memoria de la disponible en el equipo (he de comentar que no he podido lograr replicar este error).
    \end{itemize}

    A continuación un ejemplo de retornos de aceleración entre la primera y segunda versión de confBiasUpdatePar respectivamente:

    \[(0.8207472178060413, 0.14817298022125377, 0.2436479128856624, 0.27649505234475835, 0.37090513895357863, 0.3156890796375775, 0.43145311916324236, 0.29075274177467597, 0.359843417075467, 0.927856261143876, 1.333425567238517, 1.2416312895817385, 2.259662102473498, 1.809371031348172)\]

    con media: $0.7735$.

    \[(0.49370646462177686, 0.25204695669330174, 0.2545710267229255, 0.20874403815580286, 0.23218013719057562, 0.23542445644647558, 0.2920592193808883, 0.3730704569769219, 0.47244897959183674, 0.6780619111709287, 1.13201517874975, 1.8333333333333335, 1.2628692816915323, 2.6636148094109067)\]

    con media: $0.7417$.
    \\
    \\
    Como es posible observar, aunque las medias no difieren por mucho en principio, puede que esa pequeña diferencia se vuelva algo más significante conforme se trabaje con conjuntos de datos más grandes, además, hay que tener en cuenta el gasto de recursos de la segunda versión.

    \subsection{Impacto}

  \section{Pruebas y Resultados}

  \section{Conclusiones}

\end{document}